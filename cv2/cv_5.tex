%----------------------------------------------------------------------------------------
%	PACKAGES AND OTHER DOCUMENT CONFIGURATIONS
%----------------------------------------------------------------------------------------

\documentclass{resume} % Use the custom resume.cls style

    \usepackage[left=0.75in,top=0.6in,right=0.75in,bottom=0.6in]{geometry} % Document margins
    \usepackage{multicol}
    \setlength\parskip{.4\baselineskip}

    \name{Jianjin Xu} % Your name
    \address{E-mail: xujj15@mails.tsinghua.edu.cn}
    \address{Mobile:(+86)158-1058-2612}
    \address{Room 602B, Zijing \#2, Tsinghua University}
    
    \begin{document}
    
    %----------------------------------------------------------------------------------------
    %	EDUCATION SECTION
    %----------------------------------------------------------------------------------------
    
    \begin{rSection}{Education}
    
    {\bf Tsinghua University} \hfill {\em Aug. 2015 - present} \\ 
    Bachelor of Engineering(Anticipated) \\
    Department of  Computer Science \& Technology \\
    GPA: 3.14, Ranking: 96/139 \\
    Top major course grade:
    \begin{multicols}{2}
    Parallel Computation 4.0 \\
    Artificial Neural Networks 3.7 \\
    Principles \& Practice of Compiler Construction 3.7 \\
    Stochastic Mathematical Methods 3.7 \\
    Foundation of Object-Oriented Programming 3.7 \\
    Students Research Training 4.0
    \end{multicols}
    \end{rSection}
    
    %----------------------------------------------------------------------------------------
    %	WORK EXPERIENCE SECTION
    %----------------------------------------------------------------------------------------
    
    \begin{rSection}{Research Interests}
        
    Deep learning interpretation / Interactive GAN / Neural video stylization
    
    \end{rSection}

    \begin{rSection}{Research Experience}
    
    %------------------------------------------------
    
    \begin{rSubsection}{Frame Difference-Based Temporal Loss for Video Stylization}{June 2017 - }\
        \item This project proposes a simple loss function to address the temporal stability problem in video stylization (transfer the style of video into an artwork). In user study, our method performs better than existing methods. \\
        \item It will be submitted soon, in which I am the 1st auther. \\
    \end{rSubsection}
    
    \begin{rSubsection}{Neural Painter: A smart image manipulator based on simple line-drawing}{Oct. 2017 - } \
        \item This project aims at generating realistic paintings through sketches, especially the one drew by amateurs. We advanced the work of interactive GAN, improved its quality in  generation and edit. \\
        \item The demo of this project is online at 166.111.17.31:2333, which support creation and modification of anime character faces from line-drawings. Extensions to shoes and hangbags are also tested. \\
        \item I am the leader and 1st contributor.
    \end{rSubsection}
    
    \begin{rSubsection}{A simple gomoku AI based on DNN and Monte Carlo Tree Search}{Apr. 2016 - July 2016} \
    \item In the project, a gomoku dataset is collected, a policy network is trained and Monte Carlo Tree Search is implemented. \\
    \item Our simple AI is evaluated on Renju AI, showing comparative performance. \\ 
    \item This is a course project and in which I am the leader and 1st contributor.
    \end{rSubsection}

    % 实习:盒子鱼

    % 大创:可解释性

    % 社团经历

    % 实习:微软
    
    \end{rSection}
    %------------------------------------------------
    
    %----------------------------------------------------------------------------------------
    %	TECHNICAL STRENGTHS SECTION
    %----------------------------------------------------------------------------------------

    
    %----------------------------------------------------------------------------------------
    %	EXAMPLE SECTION
    %----------------------------------------------------------------------------------------
    
    %\begin{rSection}{Section Name}
    
    %Section content\ldots
    
    %\end{rSection}
    
    %----------------------------------------------------------------------------------------
    
    \end{document}
    